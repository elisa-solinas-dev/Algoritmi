\chapter{Notazione asintotica}

\section{Notazione $\Theta$\\
Limite asintoticamente stretto}
\begin{equation}\begin{aligned}
	f(n) \in \Theta(g(n)) \qquad \leftrightarrow \qquad \exists c_1, c_2, n_0: \quad \forall n \geq n_0 \quad \\
	0 \leq c_1 \cdot g(n) \leq f(n) \leq c_2 \cdot g(n)
\end{aligned}\end{equation}
\textbf{Informalmente}: Esistono due costanti $c_1, c_2$ tali che $f(n)$ possa essere rinchiusa tra $c_1 \cdot g(n)$ e $c_2 \cdot g(n)$

\section{Notazione $O$-grande\\
Limite asintotico superiore}
\begin{equation}\begin{aligned}
	f(n) \in O(g(n)) \qquad \leftrightarrow \qquad \exists c, n_0: \quad \forall n \geq n_0 \\
	0 \leq  f(n) \leq c \cdot g(n)
\end{aligned}\end{equation}
\textbf{Informalmente}: $g(n)$ è un limite superiore a $f(n)$ a meno di una costante.

\section{Notazione $\Omega$\\
Limite asintotico inferiore}
\begin{equation}\begin{aligned}
	f(n) \in O(g(n)) \qquad \leftrightarrow \qquad \exists c, n_0: \quad \forall n \geq n_0 \\
	 0 \leq c \cdot g(n) \leq  f(n)
\end{aligned}\end{equation}
\textbf{Informalmente}: $g(n)$ è un limite inferiore a $f(n)$ a meno di una costante.

\section{Notazione $o$-piccolo\\
Limite asintotico superiore non stretto}
\begin{equation}\begin{aligned}
	f(n) \in O(g(n)) \qquad \leftrightarrow \qquad \forall c \exists n_0: \quad \forall n \geq n_0 \\
	0  \leq  f(n) \leq c \cdot g(n)
\end{aligned}\end{equation}
\textbf{Informalmente}: La funzione $f(n)$ diventa insignificante rispetto a $g(n)$ quando $n$ tende a $\infty$:
\begin{equation}
	\lim_{n \rightarrow \infty} \frac{f(n)}{g(n)} = 0
\end{equation}