\chapter{Relazioni di ricorrenza}

\section{Il metodo dell’esperto}
\subsection{Relazioni lineari a partizione costante}
\begin{equation}
	T(n) = 
	\begin{cases}
		d \qquad \qquad n \leq m \leq h\\
		\sum_{1 \leq i \leq h} a_i \cdot T(n-1) + c\cdot n^{\beta} \qquad n >m
	\end{cases}
\end{equation}
Consideriamo $c > 0, \beta \geq 0, a = \sum_{1 \leq i \leq h} a_i$:
\begin{equation}
	\begin{cases}
		T(n) \in O (n^{\beta + 1}) \quad a = 1\\
		T(n) \in O (a^n \cdot n^{\beta}) \quad a \geq 2\\
	\end{cases}
\end{equation}	


\subsection{Relazioni lineari a partizione bilanciata}
\begin{equation}
	T(n) = 
	\begin{cases}
		d \qquad \qquad n = 1\\
		a \cdot T(\frac{n}{b}) + c \cdot n^{\beta} \qquad n > 1
	\end{cases}
\end{equation}
Consideriamo $a \geq 1, b \geq 2, c > 0, d, \beta \geq 0, \alpha = \frac{\log{a}}{\log{b}}$:
\begin{equation}
	\begin{cases}
		T(n) \in O (n^{\alpha}) \qquad \qquad \alpha > \beta\\
		T(n) \in O (n^{\alpha} \log{n}) \qquad  \alpha = \beta\\
		T(n) \in O (n^{\beta}) \qquad \qquad \alpha < \beta\\
	\end{cases}
\end{equation}	

\section{Svolgimento delle ricorrenze}
\subsection{Usare le sommatorie}
\begin{equation}\begin{aligned}
	\sum_{k = 1}^{n} k = \frac{1}{2} \cdot n \cdot (n+1)\\
	\sum_{k = 0}^{n} x^k = \frac{x^{n+1} - 1}{x - 1}\\
	\sum_{k = 1}^{n} \frac{1}{k} = \log_n{n} + O(1)
\end{aligned}\end{equation}


